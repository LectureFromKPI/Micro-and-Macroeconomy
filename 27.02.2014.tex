\section{Якась задача} \marginpar{\framebox{27.02.2014}}
\begin{eqnarray}
&\max U(\vx),\vx\in\mR^n\\
&D - \mt{p}\vx = 0\\
&L = u(x) +\la(D-\mp{p}\vx\\
&\dd Lx = \dd ux + \la p\\
&x^* = x^*(p,D)\\
&u^*(x^*(p,D))
\end{eqnarray}
$u^*$ - це \textbf{непряма функція корисності}. Одиниця виміру юділь (udil).
\begin{eqnarray}
&\tilde{H}=\begin{pmatrix}
0 & -\mt{p}\\
-p & H
\end{pmatrix}\\
&\det \tilde{H} = -(\mt p H^{-1} p) \det H\neq 0
\end{eqnarray}
\section{Задача №2. Економічний зміст множника Лагранджа $\la^*$}
\begin{eqnarray}
\la^*>0 \\
\dd{u^*}D = \dd{}D\cb{u^*\cb{D,p}}
\end{eqnarray}
Економічний зміст множника Лагранджа - він визначає граничну корисність грошей.
\subsection{Задача №3}
Довести, що непряма функція корисності є чимось. О боги, як мене бісить ця лекція.\\
Безкоштовних сніданків не буває.