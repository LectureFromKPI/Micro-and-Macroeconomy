%раптово малюнок
%повтикати всюду вектори та форматувати
Основні теми курсу:\marginpar{\framebox{13.02.2014}}
\begin{itemize}
\item Теорія потреб;
\item Теорія фірми; 
\item Теорія ринкових структур;
\item Макроекономічні системи;
\item Макроекономічна система Кейнса;
\end{itemize}
\section{Введення у математичні методи економічного аналізу}
\subsection{Додадньовизначенні матриці}
Розкладемо деяку функцію $f(x)$ в ряд Тейлора:
\begin{equation}
f(x+h) = f(x)+hf'(x)+h^2f''(x)
\end{equation}
Якщо $x$ точка мінімуму, то $f(x+h)-f(x)=h^2f''(x)>0$\\
$x\in\mR^n,f:\mR^n\to\mR^n$\\
$\vf\cb{\vx+\vh} = \vf\cb{\vx}+\nabla^Tf \cdot h + \vht \cdot A \cdot h$\\
Якщо $x$ точка мінімуму, то $f(x+h)-f(x) = h^T \cdot A \cdot h > 0$
\subsubsection{Властивості доданьовизначенної матриці}
\begin{teor}
Для доданьовизначенної матриці A виконується:\\
$e_i^t A e_i = a_ii$
\end{teor}
\begin{teor}
Для усіх $\la_i$, які є власними числами додатньовизначенної матриці A виконується: $\la_i>0$.
\end{teor}
\begin{proof}
$Ax=\la x$ \\
$x^T A x = \la x^T x$ \\
$\la = \cfrac{x^T A x}{x^T x} > 0$
\end{proof}
\subsubsection{Задача цариці Додони}
%ще один малюнок
Постановка задачі: Є одна шкура вівці. Порізавши її на довгу нитку, потрібно обвести максимальну по площі ділянку біля моря.
\begin{eqnarray}
\max xy\\
2\cdot x+ y  = a
\end{eqnarray}
Розв’яжемо цю задачу за допомогою фуункції Лагранджа: $L=x\cdot y +\la\cb{a-2x-y}$\\
Після диференціювання отримали систему:
\begin{equation}
\system{
\dd Lx = y - 2\la = 0\\
\dd Ly = x - \la = 0
}
\end{equation}
Отримали з неї: \\
\begin{equation}
\system{
y=2\la \\
x=\la
}
\end{equation}
Розв’яжемо отримане рівняння $2\la+2\la=4$.\\
Отримали:$\la = \cfrac a4 \Rightarrow y=\cfrac a2,x=\cfrac a4$
\subsubsection{Матрична арифметика}
$x\in\mR^n: C_{m\times p} = A_{m\times n} \cdot L_{n\times p}$\\
$(\vxt,vx)_{1\times1},(\vx,\vxt)_{n\times n}$\\
$xA$ %перекреслити
$Ax$ \\
$\phi_1(x)=C^\mathrm{T}x = C_1x_1+\ldots + C_nx_n$\\
$\dd{\phi_1}{x} = C$\\
$\phi_2  = \vxt A \vx = \cb{\vx,A\vx} = \tr \cb{\vx\vxt A}$\\
$\dd{\phi_2}x = 2A\vx$\\
$\dd{\phi(A)}{A} = \dd{\phi}{\aij}$\\
Зробити дома: $\dd{\det A}{A}=?$\\
\subsubsection{Ортогональні матриці}
Матриця $A$ називається ортогональною, якщо $AA^\mathrm{T} = I$\\
Розглянемо деякі вектори $\vy=A\vx$. При яких матрицях $A$ співпадають $||\vx|| = ||\vy||$?\\
$||\vy||^2 = \vyt \vy = \vxt A^\mathrm{T} A \vx = \vxt \vx$\\
Якщо $A$ ортогональна, то $A^{-1} = A^\mathrm{T}$\\
$\det\cb{A}=\pm 1$\\
$|\la(A)|=1$\\
$C=A\cdot B$ - добуток ортогональний також ортогональна.\\
$C^\mathrm{T} \cdot C = B^\mathrm{T} \cdot A^\mathrm{T} \cdot A \cdot B = B^\mathrm{T} \cdot B = I$\\
\begin{exs}[матриця Якобі]
$A=\begin{matrix}
\cos \phi & -\sin\phi \\
\sin\phi & \cos\phi
\end{matrix}$\\
$\det A = 1$ \\
Характеристичне рівняння: $\la^2 - 2\cos\phi \la +1$\\
$\la_\pm = \cos\phi \pm \sqrt{\cos^2\phi - 1} = -\cos\phi \pm i\sin\phi$
\end{exs}
\subsubsection{Власні числа}
Знайти екстремальну точку кривої другого порядку.\\
\begin{eqnarray}
\min \vxt A\vx \\
\vxt \vx = 1
\end{eqnarray}
Складемо функцію Лагранджа: $L = \vxt A \vx + \la\cb{1-\vxt\vx}$\\
$\dd Lx = 2A\vx -\la \cdot 2\vx =0$\\
Отримали:$A\vx = \la\vx$
\subsubsection{Функції від матриць}
$A = \mt T \begin{matrix}
\la_1 &\ldots & 0 \\
\vdots & \vdots & \vdots \\
0 & \ldots & \la_n
\end{matrix} T$ \\
$A^2 = \mt T \Lambda T \mt T \Lambda T = \mt T \Lambda^2 T$\\
$f(A) = \mt T \begin{matrix}
f(\la_1) &\ldots & 0 \\
\vdots & \vdots & \vdots \\
0 & \ldots & f(\la_n)
\end{matrix} T$\\
$e^A = \mt T \begin{matrix}
e^\la_1 &\ldots & 0 \\
\vdots & \vdots & \vdots \\
0 & \ldots & e^\la_n
\end{matrix} T$\\
$\det e^A = e^{\la_1+\ldots+\la_n}$\\
$\la_1+\ldots+\la_n = \tr A$\\
$\la_1\cdot\ldots\cdot\la_n = \det A$\\
Отримали: $\det e^A = e^{\tr A}$\\
\subsubsection{Теорема Гамільтона-Кері}
\begin{exs}
$\phi(\la) = \la^2 + k_1\la + k_2$\\
$\phi\cb{A} = A^2 + k_1 A + k_2 I \equiv 0$
\end{exs}
Через це можна виразити:\\
$A^2 = -k_1 A -k_2I$\\
$A^3 = A^2\cdot A$\\
$A^n = \alpha_1 A +\alpha_2 I$\\
$\la^n = \alpha_1 \la+\alpha_2$\\